%%%%%%%% ICML 2018 EXAMPLE LATEX SUBMISSION FILE %%%%%%%%%%%%%%%%%

\documentclass{article}

% Recommended, but optional, packages for figures and better typesetting:
\usepackage{microtype}
\usepackage{graphicx}
\usepackage{subfigure}
\usepackage{booktabs} % for professional tables
\usepackage{enumitem}

\usepackage{graphicx}
\usepackage{sourcecodepro}
\usepackage{listings}
\usepackage{amsfonts}
\usepackage{tikz}
\usepackage{tikz-qtree}
\usepackage{amsthm}
\usepackage{bm}
\usetikzlibrary{bayesnet}
\usetikzlibrary{arrows}
\usepackage{color}
\usepackage{caption}
\usepackage{subcaption}
\usetikzlibrary{backgrounds}
%\usepackage{tikz}
%\usepackage{tikz-qtree}
%\usepackage{amsthm}
%\usepackage{bm}
%\usetikzlibrary{bayesnet}
%\usetikzlibrary{arrows}

\usepackage{mathtools}% superior to amsmath
\usepackage{tikz}
\makeatletter
\newcommand\ccirc[1]{%
\mathpalette\@ccirc{#1}%
}
\newcommand\@ccirc[2]{%
\tikz[baseline=(math.base)] \node[draw,circle, inner sep=1pt] (math) {$\m@th#1#2$};%
}
\newcommand\gcirc[1]{%
\mathpalette\@gcirc{#1}%
}
\newcommand\@gcirc[2]{%
\tikz[baseline=(math.base)] \node[draw,circle, fill=gray!30, inner sep=1pt] (math) {$\m@th#1#2$};%
}
\makeatother


% hyperref makes hyperlinks in the resulting PDF.
% If your build breaks (sometimes temporarily if a hyperlink spans a page)
% please comment out the following usepackage line and replace
% \usepackage{icml2018} with \usepackage[nohyperref]{icml2018} above.
\usepackage{hyperref}

% Attempt to make hyperref and algorithmic work together better:
\newcommand{\theHalgorithm}{\arabic{algorithm}}

% Use the following line for the initial blind version submitted for review:
%\usepackage{icml2018_ift6269}

% If accepted, instead use the following line for the camera-ready submission:
\usepackage[accepted]{icml2018_ift6269}
% SLJ: -> use this for your IFT 6269 project report!

% The \icmltitle you define below is probably too long as a header.
% Therefore, a short form for the running title is supplied here:
\icmltitlerunning{Submission and Formatting Instructions for ICML 2018}

\begin{document}

\twocolumn[
\icmltitle{IFT 6269 Final Project: From Probabilistic Graphical Models to Circuits}

% It is OKAY to include author information, even for blind
% submissions: the style file will automatically remove it for you
% unless you've provided the [accepted] option to the icml2018
% package.

% List of affiliations: The first argument should be a (short)
% identifier you will use later to specify author affiliations
% Academic affiliations should list Department, University, City, Region, Country
% Industry affiliations should list Company, City, Region, Country

% You can specify symbols, otherwise they are numbered in order.
% Ideally, you should not use this facility. Affiliations will be numbered
% in order of appearance and this is the preferred way.
\icmlsetsymbol{equal}{*}

\begin{icmlauthorlist}
\icmlauthor{Breandan Considine}{socs,kast,mila}
\end{icmlauthorlist}

\icmlaffiliation{socs}{McGill University, School of Computer Science}
\icmlaffiliation{kast}{Knowledge and Software Technology Lab}
\icmlaffiliation{mila}{Mila Queb\'ec}

\icmlcorrespondingauthor{Breandan Considine}{breandan.considine@mail.mcgill.ca}

% You may provide any keywords that you
% find helpful for describing your paper; these are used to populate
% the "keywords" metadata in the PDF but will not be shown in the document
\icmlkeywords{Machine Learning, ICML}

\vskip 0.3in
]

% this must go after the closing bracket ] following \twocolumn[ ...

% This command actually creates the footnote in the first column
% listing the affiliations and the copyright notice.
% The command takes one argument, which is text to display at the start of the footnote.
% The \icmlEqualContribution command is standard text for equal contribution.
% Remove it (just {}) if you do not need this facility.

%\printAffiliationsAndNotice{}  % leave blank if no need to mention equal contribution
\printAffiliationsAndNotice{} % otherwise use the standard text.

\begin{abstract}
    Probabilistic graphical models (PGMs) are very expressive, but even approximate inference on belief networks is NP-hard.~\citep{dagum1993approximating} We can faithfully represent a large class of PGMs and their corresponding distributions as probabilistic circuits (PCs)~\citep{choi2020probabilistic}, which are capable of answering MAP queries exactly in polynomial time and are empirically tractable to calibrate. A PC is formed by taking recursive sums and products of probability functions, whose parameters may be learned via SGD or EM. Its semiring structure shares many similarities to PGMs and allows us to propagate statistical estimators like variance and higher moments using simple algebraic rules. In this work, I show how to compile tree belief networks to PCs and demonstrate their equivalence on a few toy problems.
\end{abstract}

\section{Introduction}\label{sec:intro}

A random variable (RV) is a variable which may take on values with certain probabilities, which sum to 1. A probability distribution is a distribution over one or more random variables. Syntactically, it has the following grammar:

\begin{align*}
    \mathcal{D} & \rightarrow \mathcal{U} \ldots \mathcal{N} \\
    V & \rightarrow v \sim \mathcal{D}\\
    P(V, V) & \rightarrow P(V | V)
\end{align*}

A factorized distribution is a factorization over the joint probability. It has the following denotational semantics:

\begin{align*}
    X\perp Y \mid Z \overset{\text{def}}{\longleftrightarrow} P(X,Y\mid Z) \propto P(X \mid Z)P(Y \mid Z)
\end{align*}

A directed graphical model (DGM) is a graph whose edges represent the conditional dependence between RVs.

A DGM has the following denotational semantics (todo):

% http://maximustann.github.io/mach/2015/07/06/belief-network-2/

\begin{align*}
  P(X, Y | Z) \propto P(X|Z)P(Y|Z)&\overset{\text{def}}{\longleftrightarrow}\ccirc{X}\leftarrow\gcirc{Z}\rightarrow\ccirc{Y}  \\
  P(X, Y | Z) \propto P(X|Z)P(Y|Z)&\overset{\text{def}}{\longleftrightarrow}\ccirc{X}\rightarrow\gcirc{Z}\leftarrow\ccirc{Y}  \\
  P(X, Y | Z) \propto P(X|Z)P(Y|Z)&\overset{\text{def}}{\longleftrightarrow}\ccirc{X}\rightarrow\gcirc{Z}\rightarrow\ccirc{Y} \\
%    X\perp Y \mid Z \overset{\text{def}}{\longleftrightarrow} P(X,Y\mid Z) = P(X \mid Z)P(Y \mid Z)
\end{align*}

A belief network (BN) is an acyclic DGM of the form:

\begin{equation}
    P(x_1,\ldots,x_D)=\prod_{i=1}^D P(x_i|\texttt{parents}(x_i))
\end{equation}

A path between two vertices $\ccirc{A} \ldots \ccirc{B}$ is blocked if:

\begin{enumerate}[(a)]
    \item $\ccirc{A} \ldots \rightarrow \gcirc{V} \rightarrow \ldots \ccirc{B}$ or $\ccirc{A}\ldots\leftarrow\gcirc{V}\rightarrow\ldots\ccirc{B}$.
    \item $\ccirc{A}\ldots\rightarrow \ccirc{V} \leftarrow\ldots\ccirc{B}$ and $\gcirc{?} \not\in \texttt{desc}(\ccirc{V})$.
\end{enumerate}

$\ccirc{A}$ and $\ccirc{B}$ are d-separated if all paths are blocked.

A probabilisitc circuit (PC) is a grammar over distributions:

\begin{align*}
    PC &\rightarrow v \sim \mathcal{D} \\
    PC &\rightarrow PC + PC \\
    PC &\rightarrow PC \times PC
\end{align*}



%\begin{center}
%    \begin{tabular}{cc}
%        \begin{figure}
%            \tikz{
%            \node[obs] (z) {$z$};%
%            \node[latent,above=of z,xshift=-1cm,fill] (x) {$x$}; %
%            \node[latent,above=of z,xshift=1cm] (y) {$y$}; %
%            \edge {x,y} {z}  }
%        \end{figure} & \begin{figure}
%                           \tikz{
%                           \node[obs] (z) {$z$};%
%                           \node[latent,above=of z,xshift=-1cm,fill] (x) {$x$}; %
%                           \node[latent,above=of z,xshift=1cm] (y) {$y$}; %
%                           \edge {z} {x,y}   } \end{figure}
%    \end{tabular} \\
%    $P(X,Y|Z) \propto P(Z|X,Y)P(X)P(Y)$ & $P(X,Y|Z)=P(X|Z)P(Y|Z)$
%\end{center}



\bibliography{example_paper}
\bibliographystyle{icml2018}

\end{document}